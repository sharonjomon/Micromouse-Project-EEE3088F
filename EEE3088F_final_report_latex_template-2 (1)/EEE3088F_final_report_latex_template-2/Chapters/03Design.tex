% ----------------------------------------------------
% Subsystem Design
% ----------------------------------------------------
\documentclass[class=report,11pt,crop=false]{standalone}
\input{../Style/ChapterStyle.tex}
\begin{document}
% ----------------------------------------------------
\chapter{Subsystem Design} \label{ch:design}
\vspace{-1cm}
% =====================================================
\section{Design Decisions}


\subsection{Final Design}
The following design...
% 1. Provide a description of your solution and motivate why this design was chosen.
% 2. Provide the final schematic, make sure to include:
%       - labels and component values
%       - descriptions/comments on different parts of the schematic
%       - the completed schematic page setup: title, date, revision number, author name and surname.
%       - power flags on all power connections (3V, 5V, Gnd, etc)
% 2. Provide the final PCB, make sure to include:
%       - front, back and 3D view 
\begin{figure}[h]
    \centering
    \includegraphics[width=0.5\linewidth]{EEE3088F_latex_template/Figures/signature.jpg}
    \caption{Schematic}
    \label{fig:schematic}
\end{figure}

\begin{figure}[h]
     \centering
    % First subfigure
    \begin{subfigure}[b]{0.3\linewidth}
        \includegraphics[width=\linewidth]{EEE3088F_latex_template/Figures/signature.jpg}
        \caption{Front PCB}
        \label{fig:PCB_front}
    \end{subfigure}
    \hfill % Space between the subfigures
    % Second subfigure
    \begin{subfigure}[b]{0.3\linewidth}
        \includegraphics[width=\linewidth]{EEE3088F_latex_template/Figures/signature.jpg}
        \caption{Back PCB}
        \label{fig:PCB_back}
    \end{subfigure}
    \hfill % Space between the subfigures
    % Third subfigure
    \begin{subfigure}[b]{0.3\linewidth}
        \includegraphics[width=\linewidth]{EEE3088F_latex_template/Figures/signature.jpg}
        \caption{3D PCB}
        \label{fig:PCB_3D}
    \end{subfigure}
    \caption{PCB}
    \label{fig:PCB}
\end{figure}
% =====================================================
\section{Failure Management}

\begin{table}[h]
    \centering
    \caption{CAPTION} \label{tab:failuremanagement}
    \begin{tabular}{c|l}
        \hline
        \textbf{Name} & \textbf{Description} \\
        \hline
          &  \\
          &  \\
         &  \\
         &  \\
         \hline
    \end{tabular}
\end{table}

% =====================================================
\section{System Integration and Interfacing}
To integrate the subsystem with the rest of the system ....
\\
\begin{table}[h]
  \begin{center}
    \caption{Interfacing specifications}
    \label{tab:Interfacing}
    \begin{tabular}{ >{\centering\arraybackslash}m{3cm}  m{5cm} m{7cm}}
      \hline
      \textbf{Interface} & \textbf{Description} & \textbf{Pins/Output} \\   
      \hline
      I001 & SparkFun 9DoF IMU Breakout board to STM for data transfer (SPI) & \tabitem MISO: Breakout MISO* to STM PB14
      \newline\indent\tabitem MOSI: Breakout MOSI* to STM PB15\newline\indent\tabitem SCLK: Breakout SCLK* to STM PB13\newline\indent\tabitem CS: Breakout CS* to STM PB12 (GPIO)\newline\indent\tabitem Power: Breakout 1V8-5V5* to STM 3V\newline\indent\tabitem GND: Breakout GND* to STM GND\\
      \hline
      I002 & STM to FTDI for data transfer to computer (UART) &\tabitem STM PA2(RX) to FTDI RXD \newline\indent\tabitem STM PA3(TX) to FTDI TXD \\
      \hline 
      I003 & FTDI to computer for data transfer from STM & \tabitem Assuming the FTDI is connected to a Micro B USB port (which can be used to connect to a USB cable to connect to a computer), output via cable  \\
      \hline
    \end{tabular}
  \end{center}
\end{table}

% ----------------------------------------------------
\ifstandalone
\bibliography{../Bibliography/References.bib}
\fi
\end{document}
% ----------------------------------------------------