% ----------------------------------------------------
% Requirements Analysis
% ----------------------------------------------------
\documentclass[class=report,11pt,crop=false]{standalone}
\input{../Style/ChapterStyle.tex}
\begin{document}
% ----------------------------------------------------
\chapter{Requirements Analysis} \label{ch:reqAnalysis}
\vspace{-1cm}
% ----------------------------------------------------

\section{Requirements}
The requirements for a micromouse power module are described in \autoref{tab:requirements}.
\begin{table}[h]
    \centering
    \caption{User and functional requirements of the power subsystem.}    \label{tab:requirements}
    \begin{tabular}{c|l}
        \hline
        \textbf{ Requirement ID} & \textbf{Description} \\
        \hline
         UR01 & \\
         FR01 & \\
         &  \\
         &  \\
         \hline
    \end{tabular}
\end{table}

% =====================================================
\section{Specifications}
The specifications, refined from the requirements in \autoref{tab:requirements}, for the micromouse power module are described in \autoref{tab:specifications}.
\begin{table}[h]
    \centering
    \caption{Specifications of the sensing subsystem derived from the requirements in \autoref{tab:requirements}.}    \label{tab:specifications}
    \begin{tabular}{c|l}
        \hline
        \textbf{Specification ID} & \textbf{Description} \\
        \hline
         SP01 & User \\
         &  \\
         &  \\
         \hline
    \end{tabular}
\end{table}

% =====================================================
\section{Testing Procedures}
A summary of the testing procedures detailed in \autoref{ch:atp} is given in \autoref{tab:atps_summary}.
\begin{table}[h]
    \centering
    \caption{CAPTION}
    \label{tab:atps_summary}
    \begin{tabular}{c|l}
        \hline
        \textbf{Acceptance Test ID} & \textbf{Description} \\
        \hline
         AT01 & User \\
         &  \\
         &  \\
         \hline
    \end{tabular}
\end{table}

% =====================================================
\section{Traceability Analysis}
The show how the requirements, specifications and testing procedures all link, \autoref{tab:matrix} is provided.

\begin{table}[h]
    \centering
    \caption{Requirements Traceability Matrix}
    \label{tab:matrix}
    \begin{tabular}{|c|c|c|c|}
        \hline
        \# & Requirements & Specifications  & Acceptance Test\\
        \hline
         1 & UR01 &  SP01 & AT01 \\
         2 & FR01 & SP02,SP05 & AT03 \\
         & &  & \\
    \hline
    \end{tabular}
\end{table}

\subsection{Traceability Analysis 1}
UR01 is this from which SP01, blah blah blah, can be derived. To test this AT01 is suggested because blah blah blah.

\subsection{Traceability Analysis 2}
From FR02, yadda yadda, SP02 and SP05 can be derived because blah blah blah. These can be tested through AT03 which tests yadda yadda yadda.


% ----------------------------------------------------
\ifstandalone
\bibliography{../Bibliography/References.bib}
\fi
\end{document}
% ----------------------------------------------------